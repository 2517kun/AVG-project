%!TEX program = xelatex
\documentclass{article}
\usepackage{fontspec}
\setmainfont{Times New Roman}
\usepackage{ctex}
\begin{document}
\section{rkgk1} % (fold)
\label{rkgk}

% section rkgk1 (end)
1你问我如何实现这个工程,而又不告诉我工程内部的详细信息,你让我怎么回答你?

2吾知子之所能,汝亦知我之求

1你非要这样说话么……

看着一脸严肃的她,我很难想象她在开玩笑。

2此非汝所知之

1好吧那我就不过问了……不过你现在出现在这里也只能表明情况已经到了非同一般复杂的情况了,是吧?

虽然我现在用的是疑问句,但是我其实是很清楚其中的情况的。秩序是源自于更高的生产力的体现。

2若然则此

1好吧不跟你扯这些有的没的了,你给我把门打开。

那看上去又数个繁杂的机关相互制约而封锁的大门就像是普通木板一样向外打开。

2勿讶,凡可为作之物,吾皆可以其为逆者

1这我还知道的,不过我还是对你的能力上表示惊讶啊,毕竟这看上去就像是进行了逆熵运动一样。

2否,但从宇宙借力,于无穷之未来时还耳。

1是么,总之我看看怎么解决吧。

里面并没有人们通常想象的那样,充满管道与线缆,或者说出人意料之外的,里面什么能称之为机械的都没有。虽然这个高塔能够直达对流层的顶端,但是里面却连一台电梯都没有。,只有简易的裸体以塔的内壁作为支撑盘旋而上。

当然,以现在的技术来看,高塔只是装饰。但我想那些大人物可能是难以理解这个道理的。

就像齐柏林飞艇,虽然在第一次世界大战中有着相当可观的载弹量和威慑能力,但是在金属蒙皮的战斗机面前就是纸糊的靶子。

塔建造的初衷只是提供一个特定的几何体对场进行干涉,以前是没有手段,采用大量的人力物力用合金去建造这么一个弱电立场塔。

当然,现在的话就是用一个小的立场塔去将本来平整的场极化,非要用现实中的例子的话就是一块大石头落在池塘里只能产生一个剧烈但短暂的波纹,而经过计算的数个小卵石却能生成一个打得多且持续时间长的涟漪。

而这个“卵石”的生成器,就在我的脚下。准确的说,是联通生成器的管道。

1好吧,那就一步一步排查错误所在吧。

1ανασκαφή
\begin{quote}
\textit{“信息的转录,本质上说这也是我所做的工作的实质。如果说所有的信息都可以被读取量化的话,那么相应的,一个行为也可以被压缩为信息,通过语言来进行实施。”}

\rightline{Алексеевна(我的老师)}
		
\end{quote}	

这是我的老师告诉我的,虽然听起来很浅显,但是要操作起来的话……我是在老师那里花了三年时间才掌握的,老师她还为此感到很欣慰。本人是习惯于用希腊文来表达,毕竟现在这个语言也和当年的拉丁文一样成为了一种只会在专业文章中出现的文字了,基本上没有人能够流畅地说出来。

虽然现在想起来,我根本没有为我的导师创造出任何的荣耀,但是我至少是不辱师名的。

眼前的地面一层一层地消失,而那些消失的材料企事业知识搬运到了别处。

……

在地下一米的深度,一个金属的构造阻止了持续的挖掘。

1离子切割机你带了么?

2无

1好吧,那把便携式震荡仪拿出来。

2亦无

1我平京也是服了,你这些基础的东西都不带,你到底带了什么。

只见Feynman从兜里拿出一把瑞士军刀,也不是由懂么罕见的那种,知识最普通的功能不超过7种的可以上飞机托运的那种小刀。

1我说你就带着玩意你可真是一点诚意都没有……

2汝可为吾保之。

1好吧,看来你上来就要强行闯过进取,根本没有考虑绕过安保系统的可能么。

2以其无义

好吧好吧,我算是懂你了……到时候会发生什么我可不管。

1σκληρά

恩,这句是我的老师教给我的,所以这个过程名字也是她起的。

虽然被称之为硬化,但其实是让分子固定在一起以同一频率震荡。

1加强,程度为三。

然后,展开的主刀刃开始发出炽热的红光,高频的震动在空气的畸变下变成了一种极其诡异的尖叫。

金属保护罩就像黄油一样被高温的刀片切割。

2至矣

1我不管,你给我看好周围,我不希望我作业的时候有什么东西打扰,特别是“种”

就像平京所预示的那样,很多的黑影开始想塔的方向聚集,因为整个自立建筑是出于图纸坚硬的平原上,方圆一公里没有任何建筑,所以可以明显的看出来——

对方的数量没有尽头。

就想要将地面铺满了一般,地面七分黑色,三分赤黄。

1我需要安静。

2吾知,卿勿言!

平京很轻易的解开了复杂纠缠的空气开关系统,用时钟漏洞快速地将设置的系统归零。

2汝何如

1你既然给我的工具和提示都那么差劲就给我耐心点,我这边也没歇着!

外面,可以听到不间断的阵阵风声,当然现在是出于一种不算理想的相对的盆地环境,是不可能有这种强度的自然风场的。

“种”虽然拥有实体,却没有足够的实体强度。

换言之,Feynman用自己所擅长的手段尽全力去阻止“种”靠近弱电立场塔。

平京直接徒手把最后一块束流器拆开。

1里面这是完全被破坏了啊……

束流器下面的所有仪器都像被揉成一团再恢复成原状的折纸,不论材质都整齐地被破坏。

1这可真是……就算你告诉我收到了什么破坏,我也难以修复啊,Feynman。

2复成否?

1没有,话说如果在这种状态下我是不可能给你打包票的,从情形来看我认为用概念……你有什么隐瞒我的么?

2皆否!

Feynman看上去非常愤怒,当然从根本上分析就是一种期望与现实落差的不满。

好吧,让你们见识一下我的能力吧,平京心想。

1Δαίμονας του Ισοτιμία

虽然说初步的观察来看一起的保护措施没有任何被破坏的痕迹,甚至在拆开外壳是都可一看到化学键断裂导致的短暂光弧。

一个封闭的系统,能够解释其能量损失的答案也只能是……

高纬度的非均匀性挤压么……

但是在这么小的空间里能够产生如此强的效果,当时到底发生了什么……

平京在心里诽谤。

既然是用概念层次破坏的……也就能用相应的解决方案来恢复。

“我已经知道原理,剩下的就让拉普拉斯妖来完成。”

然后,那被破坏的空间就像被冻结了一样,如同被不可兼得网兜兜住的空气,能见度却不断下降,在数秒内,以至于变成了完全不透光的黑色。

相对论允许双方互相观测到对方的行动变缓……从宏观上来说是时间膨胀的一种体现。

当然在拉普拉斯妖开始修复的那刹那,地下的时间就和外界割裂了。

里面到底是进行时间反演?还是通过调整改变常数使得局部快速收缩再重新生成?还是什么都没有发生?

平京都不知道,也不可能从任何可行的手段得知。

他只能知道完成的进度,当然这也不是实时的。

用概念将一件事情分割为人们可以忍受的部分,然后让那些劳动的概念汇报……

20\%

平京爬出那3米深,半径周长比为0.71的圆锥形下陷。

走出那中空的塔,平京试图向视野的最远方望去。

1哇,这可真是夸张,这个数量比起功效更多的只是一种形式上的美感了……

2程?

1你这边还能坚持多长时间

2厘日

1那还真是有点够呛……我看看有没有什么合适的方法来改变一下曲率吧……

2勿妄

1好吧,我喝听啤酒……

平京从门口的塑料袋里拿出了一听TzingTao,单手开罐然后直接灌下了一大口。

1啤酒已经温了啊……这样的话时间就难熬了。

40\%

1你就坚持不用热兵器来解决问题么?非要拿那把陌刀么,我当然知道这玩意的效率相对很好,但在电驱动的夺冠……

2毋须多言。

Feynman说罢,就直接拿出了一直庞然大物,外表用高渗碳钢板保护,虽然有超过5米长,其洁面棉结缺不超过一平方米。

1哦,终于拿出了一些有看头的东西了。

平京仰头将铝罐里的发酵液一饮而尽。

2吾之固亘古而难移,非尔之可及

虽然其心态如加农炮,炮声亦震耳欲聋,可见光与烟与火,推进却乏燃素与酸素。

1真是不严谨φράγμα ήχου

导体在洛伦兹力和延时电路的促进下机械波的屏障如同无物,就算没有了自旋依旧壁纸地飞翔无限远的找落点直到弹体完全消耗或蒸发。

60\%

1好么,你的逆反心理也太严重了一点……不过这更像是线攻击而不是面的覆盖……

2烦了

1哈?

2你运算的最短时间是多少?

1大概五分钟吧……

2一直以准确运算而获得名声的平京你怎么会说出“大约”这个词?真是可笑。

1别虚张声势了,明明是你这边精力不够了练大脑多余的运算都关闭了,我又不能帮你战斗,只能辅助你不出错罢了。如果不是那一层的声音的隔离……炮膛的机械波早就把你的耳膜真出血量。

1我说的五分钟,是300秒,误差是小于0.05秒的

1说实话……你到底还能坚持多久……

2不少于450秒

1说实话。

2……378秒。

1我让你说实话,是因为我要做出正确的评估,而不是让你用虚假的数字安慰我。

2从现在算起……不到三分钟。

1我就知道“凡是可能出错的事必定会出错”这玩意永远是戏剧性的人生的原因

80\%

距离Feynman完全无抵抗能力还有不到两分钟。

1恢复的演算加速已经是吃力不讨好的事情……那么我看看有没有什么能够帮你进行战斗的辅助……

……


bm哇,这可真是夸张。就算外科理论上是不完备的,也不至于直接将其装化为这么一大坨不稳定的能量吧……

阴天,就算高塔已经没过层云和积云,光电效应所激发的强烈橙光也穿透了这本应该难以穿透的云层,并如同晴天的阳光般刺眼。

但远处的那个观测者仍然用眼睛直接看向发光源,没有对灼伤失明的恐惧而直接注视着那边。

bm是时候回去了。

……

2你这真是夸张

1这可不用你说,我比谁都清楚。

将自身加入到运算之中,然后让与演算相关的外部仪器超功率运行……结果就是整个与仪器相连的高塔的残骸本身也被消耗,以真空能量为势能定义基准进行毫无限制的能量释放。

这是一个丝毫不考虑实际的,完全是人们口中已经没有人性的科学家才会想到的行为。

既然一件事情客观上无法避免,就从主观上避免就行了。

于是另一个被分割的时间流速被创造了出来,就算外面有着多么强的能量释放,也会因为时间流速的不同而只能略过其表面,因为没有强引力的约束,这种光芒也不可能向黑洞那样持续的释放γ射线。

1管那么多干嘛,你不就是损失了一些兵器么?而且这些对你来说只不过是一天的快餐能量的副产物。

2我可不是这个意思,从”同盟储备“去接到的能量终究是要偿还的……你进行了一个如此迅速的熵增过程……我这边也很难交代的……

1我可不会在乎这种东西,只要能有净价值,那就值得去做。

2我可是羡慕你这种纯粹的人不得了啊。

1别这样,你也是男的过来,很多事情也不是一时半会儿能够解释清楚的,学姐。

1Фрида М Алэквив


\textcircled{20}




\end{document}
