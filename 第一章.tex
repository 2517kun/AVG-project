\chapter[first chapter]{第一章}
\centerline{2038年}
缓缓从舒服的座位里醒来,并将坐姿调整为端坐,让自己适应车厢内的光线。

虽然没有任何可被感知的震动和定向的空气流动,但是从车厢外的风声和不断向后刷新的地表又无可置疑地表明我在移动。

列车的高速移动使得进出的景色都糊成一片,远处的也过于渺小而难以辨析。

但其实就算这亚音速的列车能够停下来让像我这样的乘客欣赏沿途景色,我也会感觉到难以辨析。

毕竟现在刚过完年,平均超过一尺的雪将整个地表覆盖。将这个没有车站的荒凉村庄覆盖。

这样的雪,在这边已经算小雪的了。这里,离我的老家也没有多远。

我的故乡,或者说是老家,在边罗刹,至于以前叫什么,没有人告诉过我,我也对此不感兴趣。

我的故乡是一个无趣的地方。

是的,非常无趣。四周出了自己居住的小镇和链接的沥青/混凝土混合路之外,就是树林。

黑压压的树林就算是没有树叶雪铺土地的严冬,也只能看到不到百米的距离。

一层又一层黑色的树干,只有少量的阳光透过顶部的枝干照射进来,如果没有任何准备就进入森林的话基本也不会有人能再见到那鲁莽者了。

这也是为什么我情愿去阅读自己从家里旧书架上顺下来的《资本论》然后在高速列车的软座上睡觉也不想去看这窗外故乡的景色。

我不认为这是景色,这只是外界的环境罢了。

当然除了伏特加和废弃的军用品堆砌的日常,童年还是有些许值得回忆的东西的。

还记得那时用极低的价格买些许用西里尔文字写着的罐头,那些罐头用非常结实的铁皮装着,就算是用专业的开罐器也很难轻易地打开,就算是冻上了也不会膨胀。里面大多是鱼子酱和一些别的水产。在冬天直接走过黑龙江,和那边的人换一些印着同一个人头像的纸币。

“不要红色的,棕色的多给几张。”

这是当时的经验,还是那些鄂伦春族的人告诉我的。

当年春节时,还记得自己迷失在数不清的人流之中,他们穿着深色的羽绒服,我当时什么都看不到,那些冰灯什么的我只能够从照片之中了解到其样子。

至于其他的,埋藏在记忆的深处,混杂成一团。

这种记忆,就算忘却也并不足惜。

列车继续向前行驶,一个乘务员推着一个小车零售宵夜。

?“一份煎饺,加热就不用了”

乘务员:“好的,一共是200元。现金还是刷卡?”

?“刷卡就行了”

乘务员:“对不起这位乘客,刷卡的话是使用哪个名字?您的信用卡下游……”

平京:“用‘平京’就好”

制动阀与铁轨接触,发出了一段持续且高亢的噪音。

列车已经到站,透风的车站虽然对我不算什么太大的问题,但是还是让刚从温暖的车厢内部出来的我轻微的打了个寒颤。

我回来了,北育。

这里依然是华山以北最大的城市虽然还有大部分的地方被废弃。曾经这个城市是位于两个山脉交叉处东南方的平原地带,后来因为地质学变动而实质上废弃了南部的城市,沿着燕山山脚向黏菌一样附着在上面,虽然我的比喻并不恰当,但基本上感觉是这样没有问题的。

这里,有着全人类最好的教育,最棒的医疗与物流系统。虽然才度过“灾难”后不到八年,却完全看不到一点收到过那样恐怖天灾的残留。

至少在市中心是这样的。

这种发达城市的生活成本,对我而言也是理所当然不能承受的。

住所,也当然是在北育的郊外。

火车站外的街边停满了那些没有牌照的出租车,每一个都在用尽全力招揽。

但是那些司机是无理由也不可能把我送到住所的,至于原因那就复杂了。

走在市中心的街道上,街边的路灯的橙黄连成线,随着道路延伸到远方,

现在天色已晚,但是街边那些小吃摊还在营业。用着那种脸颊的塑料布遮挡高楼的空调水,随意在并不能称之为干净的地面上没有任何规律地摆放着白色的八仙桌和可以叠放的椅子,烤串和麻辣烫飘着水蒸气和烟尘的摊位前还聚集着大量的食客。

这里离故乡还有五百里,这里离故乡还有五百里。

这诚然是个玩笑,我这么说只是表明这里和我的故乡没有任何的联系。

这里只是我现在的住所,我的故乡早已在震荡之中被磨平,然后被遗忘了。

我在这里已经断断续续生活三年了,从身无分文到自己能够经营一家小企业并能够有所盈余了。

总之,就算这里离我的住所没有五百里,还是有十多公里的路程要走的。

……

平京:“呼呼呼……累死老子了。”

虽然一路都是人行道妈蛋这崎岖不平的地面走起来和登山没什么区别,全程下来的运动量和等距离的竞走差不多。

总之,我回来了。

在我面前的,是一栋只有七层楼高的自立建筑。

听说是很有年代的那种老楼。只是刚好在灾难前做了加固延寿处理,而且周围的地基也相对比较的坚实,缺乏地下水层,总而言之这个建筑逃过一劫。

后来经过了几次转手,这栋楼就落在了我的手里,至少大部分的楼层是这样的。

一到五层全部买下来用于储存,六层买下了剩下的房间打通隔断用作居住+办公的场所。

虽然还有好几间没有买下来,但也是只有产权的那种空户,应该没有什么人居住的。

当然就算有人也和我没什么关系,毕竟这几年下来我也没有见到过有别的人进出这里。

用特殊的钥匙打开第一层房贷门,一股强烈的灰尘和臭氧混合的味道就直冲嗅觉感受器。

平京:“果不其然啊,虽然说所有散热都换成了水冷并将那些镂空处用绝缘胶带固定,电路板的漏电问题还是很严重啊……”

而且还很热,那种空气中干燥的炎热。

而且还要考虑新的通风解决方案了……要不是基建都完全毁没了,我都想在北极圈租个屋子去架设服务器了。

这些服务器要换主板的话又会是一笔巨大的开支,如果将企业及服务提价的话今年是不可能了,毕竟上个月刚根据通货膨胀率刚调整了价格,如果再调整肯定会被骂的很惨。

其实被骂成什么样都无所谓,但是客户会大幅流失,更实际的是会减少收入。

说白了就会变得生活拮据。

算了,用吸尘器把灰尘给处理了来的更现实一些。

平京:“今天我记得是五级风吧,开窗换个气应该没啥问题。”

然后我就后悔了。

打开窗户的同时,冷空气夹杂着落叶的残渣涌入这已经足够脏乱的机房。空中本来应该自由落下的那些灰尘土壤糊我一脸。

没错,就是糊我一脸,感觉从鼻腔到牙龈舌头上都有灰尘。

平京:“咳咳……是哪个脑残想到去开窗户的”

虽然我的呼吸道受到了摧残,但那巨量的灰尘也至少吹散了三分之二。

戴上那双三层涤纶的防静电手套,把手持式吸尘器通上电,调成最高档。

人要长教训,戴上了油滤芯的防毒面具。

总结一下

快半个月没操纵这些玩意了,差点拆坏一个机箱的面板。

结果是好的,从一层到五层所有的服务器和高密度储存磁带机与高速次口径迭代读写装置都得到了妥善的清理。虽然不像刚购入时那样干净,但也算是对得起“高科技”这个词了。

平京:“我是不是应该买一个除尘机器人,或者静电吸尘装置……”
如何学会停止恐惧并爱上灰尘?

我想不可能的,这种东西只会将我弄得身心俱疲。

爬上六层,回到自己在北育唯一的窝。

虽然将两套房的隔断打通了,但因为最近数据的暴增,出了用于生活的最基础空间,基本上都堆满了那些必要但对电量需求不太大的中续器和部分的raid1结构磁带储存。

虽然这些器械不会造成大量的漏电和散热导致的灰尘,但过大的体积和磁带读写的机械运作声音还是会让初次进来的人感觉难以适应。

至于我?反正我都适应了,就算是那些叽哩哇啦的针头碰撞和轨道缺乏润滑的刺耳声音都早已习惯。

不习惯还能怎样,不是照样活着么。

……

这就是我的日常生活,就这样重复了上千天得日常,基本上就是不断扩展的日常。

太阳下无新鲜事,虽然还算过得去,但是这种日子十个人我想都会认为是索然无趣的。

这种个人抱怨不提也罢。

网络无穷大,而所蕴藏的可能性我相信也是无穷大的。

我现在的住所,也是“公司”的唯一门面——Peace ISLE有限责任公司。

与其说是公司,不如说是我一个人的工坊,唯一出我以外一个雇员还是在网上联络办公汇款的那种协议工,到现在我连对方长什么样都不知道。不过虽然不是明确的契约工,但是对方的工作质量还是很让人满意的。

甚至是完美的让人感到可疑。

比如说,本来数据库逻辑是用OVERLAP\#编写的,但是对方还是用Haskell+Lisp给我重构了一遍数据库,结果是运行大幅提速并且服务器功率下降了一半。

当然也造成了一个更加恐怖的后果,这种反人类语言写出来的东西我基本读不懂,能维护的人也就只有那个“对方”了。

出于方便我的叙述,我会用对方的联络化名“美香紙”作为称呼方式。

毕竟她(这个人称代词是她在聊天中确认的)的支付账号的名字是一堆乱码,我也不知道她是用何种方式做到的。

幸运的是,她对于薪水的要求不高,基本上是一个难得的人才。

当然,她的严谨也只限于工作方面,在聊天的时候基本上不会说什么有用的东西,从最近的娱乐新闻到那些我不去用定向搜索就不可能知道的历史上的故事。

扯了这么多,这个小霸王级别的网站终于登陆成功了。

这个网站,说是网站,但也只能用telnet才能登陆,虽然现在的telnet能够支持图片和一些互动插件,但说实话,这种网站还能存在也是一个奇迹。一个没有任何美观可言的BBS。

当年,在边罗刹的时候,那边并没有什么先进的个人电脑,我用的那一台还是从废品堆里攒出来的,还记得当年用真空管接在电源处。现在想想,这种行为毫无意义。

时代在变,那种先进的网页基本没有一个是能够让我的那个“电脑”使用的,毕竟那时候就已经插件特效满天飞的网页是完全没法在那种纯文字浏览器中显示的。

登陆我的id:晴空波澜,密码:***

通过论坛助手,看到了不下十条的回复。

里面除了一两条没有什么营养的支持/反对的内容,全部是来自于一个叫做“默示AP”的人的回复和反驳。

基本上,是关于经济方面的。

我是倾向于古典经济理论模型的,而默示AP更多的是使用纳什均衡的博弈论,更倾向于“必然出现的偶然”

比如我在一个帖子中宣称股票因为所有的持有人都是理论上的信息完备取得者,没有人能够在股市中稳定而持续性的获益。

她就反驳我说在数次的巧合后持有者会对市场中曾正确选择的买卖者的行为产生盲从,从而被反向利用而达到持续性获益的目的。

我是难以反驳她精心在这种全文字的论坛上用ascii符做的图标,还有那些大把的论文引用。

我是没有点进去那些论文的链接,并不是因为我的时间紧迫,只是单纯由于那些论文所存在的资料库收费对我而言是难以接受的。

是的,一个字数不超过3000的总结性论文,底下的引用页超过4张纸的那种,收费都近乎达到了一块钱一个字的地步。

我想这不是因为洛阳纸贵这样的缘由。

我曾手动关注过一段时间默示AP的帖子,都是那种经济上的回复,用那些稍微有年代的经济诈骗。比如说2020年的那场复杂的融资诈骗、2008年的一个庞氏骗局、2030年的保险欺骗、2034年的身份信用窃取……

总结是:基于那些天真的以为进入股市/期货/基金和企业债券就能大赚一笔的人泼上冷水……对于他的话更多的是泼上一盆液氦,还是那种不到一开的液氦,从头上泼下来之后还会顺着毛细在爬回头顶的那种严寒。

反正我的语文老师早就不知道死到哪里去了,我这样不恰当的比喻应该能被人接受吧。

以前我还会和他有一些激烈的争辩,一天能够相互回帖战个上千层。后来到时没有过这么激烈的论战了,我和他我想更多的是因为那些论战得到了了解。

毕竟直到的东西就没有必要再去重复了。

说白了,就是感受到了自己的有限。

有的时候一些想说的话就是停留在脑干之中不能向下传达,与其说是缺乏逻辑更不如说是缺乏对于文学上辞藻与运用的熟练度。

平京:“终归中文不是我的母语,我能说成这样让那些中国土生土长的人听懂就已经很不错了。”

我时常这样安慰我自己。

所以,我基本上说话是比较少的,更多的是对话的倾听者而不是参与者。

论坛上的很多人都说我很客观,其实这也是一个无可奈何的结局。

我现在更多的是发起讨论,比如“北育的教育为基础的经济结构是否是可持续性的?”,“如何用最短的时间完成教育大纲?”,“如何聊天儿不显得很蠢?”

当然第三个蠢问题是我用来对付美香紙的,我并不会在这种低级的地方骗人。

……


虽然大家都说北育没有春天,只有冬天和夏天的突然转变,但这对我没有什么影响。

现在还没到三月,外面行道边的雪堆还没有化干净,在街上走的话一股冷空气就能让人打个寒噤。

不弱就算是这样的天气,我也没有启用独立制暖的小心电锅炉。

这种电锅炉虽然对于电费的增长来说忽略不计,我不开的原因仅仅是因为没有任何的必要。


虽然大家都说北育没有春天,只有冬天和夏天的突然转变,但这对我没有什么阴险。

现在还没到三月,外面行道边的雪堆还没有化干净,在街上走的话一股冷空气就能让人打个寒噤。

不弱就算是这样的天气,我也没有启用独立之暖的小心电锅炉。

这种电锅炉虽然对于电费的增长来说忽略不计,我不开的原因仅仅是因为没有任何的必要。

在我的卧室里放置着好几打的不间断电源服务装置,用于给下面一到五层的那些宝贵机器稳定的电压。

虽然说我的房间里放置着这么多的稳压电源,但如果真的断电了的话,整个储备电能撑死了也就能支撑个一刻钟,然后所有的服务器就会理所当然的宕机,重新架构Linux环境还得花费至少一个半小时的时间。

别问我这么理想的时间和后果是如何产生的,反正不是计算出来的。

总而言之,为了防止火灾而分布在我房间四周的稳压电源就像一个个暖手宝那样,虽然单体的温度并不算夸张,但多个这种供暖器已经改变了整个房间的对流方式,让我提前感受到了立春之后北育的温度。

在这种烦躁的气氛中,我不断的刷新邮箱客户端,这不光是为了工作,也是排解最低等级的无聊的一种手段。

“叮咚”

这个声音并不是门铃,其实我也根本没有安装真正意义上的门铃于我的房门,唯一一个具有体型房主有客人要求进来的工具是一个捡回来的敲门器。

虽然是捡回来的,但看上去材质又轻又硬,我很是满意。

话跑远了,这个声音其实是我的私人邮箱的提示音。一般情况下知道这个邮箱人数的不会超过你手指头的个数。

“from:陈整蚕中学行政办事处;to:平京先生”

“内容略”

其实这种充满了恭维话的新建基本上只要去读段首句就可以明白整个信件内容的框架了,甚至可以说直接看第一段的最后一句话就足够了。

以前这个中学给我发邮件都是通过工作邮箱接受的,也不知道是哪个损友将我的私人邮箱地址给抖了出来。

算了,这种事情计较也没有太大的意义,反正我现在的私人邮箱出了撑死了一周才会又一个有用的邮件,其他的都是当年注册工具时留下的邮件推送。

其实这个邮件就是“诚挚”地邀请我去执教,给那帮熊孩子。

我算算,现在是二月末,离中考也就不到四个月的时间……也是意料之外而又情理之中吧。

说白了就是一群根本没有什么上进心的学生需要提高学校的升学率,而这个学校是文科全市数一数二,理科教育基本只有基础教育大纲水平的中学。

我想,我可能就是那种类似于“外援”的角色了。

%1“选择去任教”

平京:“恩,看来时薪给的还是挺想模像样的么”

毕竟公立学校就是不一样,教师工资不用上税不说,好对项目还有实在的福利政策。

虽然我到时候也称四十一个外聘教师,但是我还是能够享受到工资税务的减免还有在外非教学工作的税务免除。

\textit{“尊敬的行政办事处人员:\\
\centerline{我需要更详细的信息平京}\\
\rightline{2038.2.28}”}

对方的回复到时出人意料的及时,完全少了那种国有企业常见的缓慢和臃肿……这让人不由的怀疑对方是否也和我一样无所事事。

\textit{“平京先生:我很荣幸能收到您的回复,我们只要求您能够每周对初三和高三的学生进行数学物理化学和部分地理历史的辅导,每周大约15个小时即可,多出来的授课时间按照合同的双倍进行结算。您所有的教学要求都挥霍的满足,这只因为本校相信您的教学水平能够带领本校学生在一次走向辉煌\\
\rightline{蚕行政管理处 江三峰}\\
\rightline{038.2.28”}}

平京:“……还真是丝毫不掩饰自身的不足啊……”

反正与其闲着坐在这个旋转座椅上无聊到发毛,还不如主动找点事干。

“明天我会到你们陈整蚕那里,至于上课的时间我来定。”

这封邮件我完全没有按照标准的格式来书写,更像是写了一张便签。

当然这并不重要,因为现在我是话语的掌握方,依我看我就算只回复一句“可以”都是能够让对方欣然接受的。

好吧,我将再一次为人师表了,虽然我其实并不喜欢那种感受就是了。

虽然讲台只比地面高了不到一尺,但是视角的转变使得我看学生的时候会感觉更加的闹心。

那些拿着巴掌大小的课本和文具袋来掩藏私人的个人终端的,在完全没有什么遮拦的课桌储存空间里吃着面包火腿的。

这还是比不上吃韭菜辣面筋的那种奇葩,我记得那个学生只要一撕开包装,整个教室都是那种廉价的味道。

他倒是吃的挺香,但闻起来并不是那么美味就是了。

话说跑了。

总之期望那些近乎于三年都在学校里划水的人能够在这么几个月跑去双休日和法定假日撑死了两个月的学习就能拿到优评级的学生,脑袋一定是被门缝夹了,还是液压门的门缝。

……

虽然这个学校同意我穿便装去任教,但是我还是穿上了白色的衬衫。说实话刚从所料包装拆开的时候整个衣服就跟用过半把年的抹布那样硬。

这种衣服穿在身上有点硌得慌,尤其是这种掺着20\%涤纶的。

不过幸好现在还是初春未到,冬天已经消去的日子。

尽管走在崎岖不欧尼的水泥路上让人略微发汗,但时不时吹过的冷空气再一次让我冷却下来。

背着我的个行李袋,向着在数公里外的公交车终点站走去。

我现在在住在北育南,就是一片未开发的地方的样子,除了我工作兼生活用的房屋外,那里目之所及就没有几个是完整的建筑。

往东南方向走个十几公里就可以看到参差不齐的海岸线,记着地理上曾经这里的海拔在43.5米为平均,现在我那里因为“灾难”的缘故整个地心向上隆起了近乎30米,然后地面高度随着不断靠近东南边的隶直海湾下降,然后再深入海岸半百千米后大陆架变成大陆坡……

当然,简而言之面目全非就是了。

那些被划分为“废弃”的楼房里面还有很多的人住在里面,有世世代代都住在这里不舍的离开的人,也有做着被强迫拆迁敲一笔巨资的人,更有父母去北边拼死拼活打工而留下来的儿童。

完全没有什么清理,但比起尘埃更多的是那种质量更大的水泥渣滓。

路旁还有那些被矩形混凝土板压坏的车辆,当然这些车也只剩下车架子了,那些有价值的和可能有价值的都被别人掏空了。

本来就已经四分五裂的路面在不均匀的地质变化下也变得难以辨析了。

走向下一个坡道,整个坡道让人们难以想象这里曾经是平原。

边上的商品楼想多米诺骨牌那样倒下,甚至可以看到楼房外板材与地基之间完美的切割。

这难以称之为道路的路边隔三差五还有这那种买东西的小摊位,大部分都是贩卖零件的,他们的摊位座椅边上十有八九都有着一个金属探测器。

继续大步向前走,无视着那些重复出现的挂满流锈的房屋和拦腰截断的立交桥,跨过分割作为南北区别的检疫隔离门。

当然在抗议和效果双重促进下这个检疫站组成的长城算是废弃了,整个检疫站也只是保留着门框的栅栏一类的东西。

过了检疫站没过多远,整个状况都发生了改变。

路面就像有一道贯穿东西的分割线,分割线的南边是废墟,而北边却是整洁的道路和楼宇。

这种令人瞠目结舌的反差,要归功于弱电立场塔。

在这里都可以清晰地看到,那个突破天际的如同电视信号塔一样的装置。

记得刚要去建造的时候,听新闻说有很多人反对这个巨型装置的建立,大概的理由和当年一堆法国艺人拒绝埃菲尔铁塔的建成的里有差不多,只不过是加上了更多能让民众相信的伪科学内容。

比如说这么大的装置肯定会将北育(虽然当年并不叫这个名字,但是我并不记得了)地震带上的平衡破坏;阻碍北育的地下水脉;影响北风的流速造成污染物堆积;产生的电磁场会对人健康有害……当然我记得最扯的是有一个不知道什么地方的小报纸宣称这个建在北育北的高耸入云的装置会影响北育的龙脉。

这种东西就当笑话听,当时我还在想,不就是一个高塔么,至于产生这么大的反应么?

后来,我第一次到北育,坐高速列车从隧道穿过山峦后,看到的第一个建筑,就是弱电立场塔。

这个机械仪器堆砌而成的高塔看资料有超过13千米高,塔顶的反应装置会将反应点圆半径2公里内的空气变成那种类似于膏状的东西并不断反应发出紫色光。

开始时我曾认为是等离子态的空气,但实际上是弱电在简并能量的不可避免的外泄。

我并没有去对这个装置做出过深入的研究,所以也不想对这个高塔的科研作用做出任何评价。

后来,我记得在灾难的时候,这个塔在不知道谁的操控下产生了人为的大量能量外泄,形成了一层物理影响上的“反介入/区域拒止”,削减、消除灾难中带来的影响与冲击波。

当然,这种超过当时理论水平的科技也是有作用范围的。就结果而言,也只能保护到这里,就是我脚踩的这个边际。

沿着平整的人行道走了几百米,就是公共交通的终点站了。

等了不到三分钟,车辆一如既往分秒不差地停到了车站口。

坐上空无一人的巴士,巴士平稳加速到了60英里每小时。

……

教室里安静得吓人。

这种情况对于那些在学校混了三年的老油条学生来说并不正常。

虽然现实中的确在任何老师的第一节课学生都会老老实实地出席上课,但是很多老师的第二节课学生就会停止严肃了。

对于我而言课上大闹基本也不会特别多,毕竟我根本不在乎学生在下面是睡觉还是操作个人终端打发时间。


我可以让那些听我讲课的人的成绩提高,甚至可以无视那些不去学习的家伙。

这就是我当教师上的自信。

总之,从头开始确认吧。

平京“我记得虽然之前你们没有上过我的课程,但是我有要求你们完成一个小册子大小的理科综合检测。”

平京“当然这玩意开卷做和闭卷做不会有多大区别,甚至公开讨论我都不会加以阻止反对。总之,写了的同学就传上来吧,如果没写的就自己留着收藏吧。”

总共小三十多号人的班级只收上来而是份出头,我已经很满意了。

平京“这一沓你们写的东西我大概会花十分钟来评估,当然不会有准确的百分制结果,最多是以五分制来进行一个模糊的评估。”

平京“这分数是给我看的,你们看了也没有什么区别。有什么问题么。”

同学A“那个,老师,我们怎么称呼你,还有这十分钟我们干什么?”

平京“额……你们叫我平老师就行,当然如果直呼我命“平京”我也不会介意,至于这十分钟……”

同学B“自习行么老师?”

这个不懂事的学生突然插话,让我产生了一丝不悦。

平京“我就在刚才那个瞬间把自习的可能性否定掉了,这样吧,我一边批改,一边给你们介绍一下你们将面对的差不多一个学期我会也至少会交给你们的内容。”

平京“我虽然还算年轻,但说白了是一个返聘老师,别想着贿赂我,就算贿赂也不会有什么好处”

恩……不行啊,虽然有少数称得上是拔尖的,但大部分都是低分飘过。


这个叫寇浩宕的学生经典力学部分我是认可了,但是在基础量子电动力学和统计量子物理学上基本就是得了一个零蛋,甚至环弦入门完全就是叫了个白卷。

平京“我就不明白了,这些内容虽然听着好像很可怕,但是只要把一百多年前那帮学者总结的公式套用一下就完了,你们却能跪一片。”

这已经算是比较好了的,还有更多的学生连数学i的内容都没有完全吃透,这样的话后面的“微观化学与键”“有机化学1北育版”什么的完全就和猴子做的选择题没有区别。

平京“麦克斯韦方程组我看也不比背平假名和片假名之间的对应难,你们怎么就没几个写对的……当然这是我个人的牢骚,你们不想听可以趴着也可以。”

当然,我觉得经历了点击训练的猴子可能还会有更高的正确率。

有些学生后面的“经典相对论导论”甚至“环弦入门”有着看得过去的成绩,但是前面的错误也不少,完全就是不扎实而求助于网络的学生。

平京“对了,校长跟我说要管学生不去玩个人终端,我就在这里声明一下,被巡视的管理人员没收我是不会保护你们的。”

这里还有一个学生,虽然答案完全正确,但是就算是第一题都有缺少必要步骤的情况……虽然我很欣赏她,但是很可惜这并不能得分。

我用“她”作为人称代词,是因为她在落款处写着“诺唯”二字。

平京“恩,你们果然和我预想的一样……”

平京“烂。”

平京“给我上交的23份答案中,89\%的傅里叶变换需要重学;76\%的微观有机化学也是一样;65\%的量子电动力学……这个方面我倒是觉得挺意外的不错;相对论一半吧,看来大家科普都还不错;然后还有大概八九个个人的基础还是一团糟,我就不在课上单独讲了,我会直接把复习资料发到你们的个人终端上自己啃去。”

平京“至于那么几个根本没有交的,我就当没有你们这样的学生存在,你们爱来不来,我都不会去管的。”

平京“说实话这个班里竟然有优秀的学生,我很意外。山神同学,你有什么感言么。”

山神“我并没有什么感想。”

平京“我建议你至少说一点,至少这样遭受学校霸凌的可能性会小很多。”

山神“我只觉得学习这种东西很有趣,仅此而已。”

平京“你能说出这个话引子就足够了,关于教学大纲我已经发给每个人的个人终端里就不做累述了,我就试着提起你们学习的兴趣吧。”

当然,后面的句子基本上是我昨天现找的。

平京“一个有着完备的逻辑思维训练的人,在事业的操纵上更容易获得成功。”

平京“换言之,如果你们的成绩能让我满意的话,肯定在自己未来的事业上有所作为。”

平京“当然比如说量子统计学和定向泛积分可以让你所有的思维竞技游戏更容易生出,比如国标麻将之类的……当然不要试着在赌场去尝试,经典的概率论就能把你打的体无完肤的,毕竟就算公平竞争你也没有赌场有钱。”

我就知道,没有人在听。

无所谓了,这是第一节介绍可,我根本不想像那些小学教师那样千方百计地吸引学生的注意。

下课铃声响了,学生纷纷走出教室。

虽然这姑且还算是义务教育的一部分,但这里的教课方式是大学的那种按照科目分配教师的。

对我的好处是免去了找教室的奔波,也不算坏。

平京“诺唯同学在么,我希望你能留下来一下。”

那些无关的学生都离开了教室,只剩下一个端坐在书桌前的紫发少女。

平京“你就是诺唯同学对吧。”

诺唯“是的,我就是。”

平京“我看了你的答案,从结果上我是很满意的,但是……”

我犹豫了一下,而那一丝不苟地坐待有可折叠班子的椅子上的诺唯困惑地歪着头。

平京“你所书写的答案全部都缺少步骤,虽然选择题我管不了你,但是占60\%权重的叙述题你缺少太多。”

平京“虽然你的思维逻辑上有着数学家的影子,但是那些判卷子的不是行尸走肉就是冰冷的机器。所以我要求你按照逻辑学清清楚楚地把过程写详尽。”

平京“你一不是伽罗瓦二不是PD费马”

平京“虽然这里考的是数学物理化学,但卷子不是实验科学,不需要提出那些旧的经验所不能支持的现象,所以我觉得亚里士多德的三段论就基本够用了。”

平京“我可能会在上课向你提问,我希望你能够到时候积极回答,听明白了么?”

诺唯似是而非地点了点头,我不知道这是真正的肯定还是和日本人在听别人说话时发出的“はいい”那样只是表示“我在听”的表示行为。

平京“好吧,你该干嘛干嘛去吧”

然后是内容的重复,但是下一个班的学生并没有什么特别的地方,没有什么值得我眼前一亮的学生也没有那种查到让我印象深刻的卷子。

就这么又教了两节课,把这一个年级所有到场的学生都教了一遍,我的任务也完成了。

礼节性地在校门口登记,作为“我曾经存在于此”的证据之类的行为。

离开校园,太阳已经存在与地球的另一边,虽然春天即将来到,但是现在还在春分日之前,夜晚总比白天长上那么一截。

坐在有着极低底盘的公交车,原路返回。

虽然说南区是很落后,但我并不觉得很乱。

人们会在日光下很谨慎,在月光下为所欲为。

所以才有了什么“月相”对人的影响什么的。

当然,这玩意已经不会有任何作用了。

只要你能看到天空中的那环带之外的唯一的卫星,它就一定是红色的。

以前有人形容月亮是天空中太阳的收敛的样子。

当然那是诗歌的描述方法,我认为其可视大小就是伸展自己的手臂,让后和自己的大拇指第一关节所占的么视野面积差不多,有的时候万里无云可能会感觉上更大一点。

现在,“狡黠的月亮”是不存在的。

那东西只是一个巨大,表面有者难计其数的坑洼甚至都不太成球形。

因为当年的“灾难”整个卫星的表面都是融化了的呈岩浆状发出红光,卫星岩层高温分解出来的二氧化碳包裹着这颗如同地狱般的卫星,眯起眼睛甚至可以看到卫星的光晕。

虽然这卫星离地球如此接近并体积巨大,但大家就当是习惯了一样,对天空中此等不平凡熟视无睹。

不过也是,那些抬头的人的工作都是由低头的人完成的。

就算古希腊有着在深邃的哲学遗产,值得那些大师豪杰当做文艺复兴的范本,也存在着远超过公民数量的外邦人和奴隶。

总之,这个卫星只是“挂在天空”罢了。

回到一片寂静的住所,走了不到十步就一头倒在床上。

平京“好累”

虽然我尽可能减少与那帮一窍不通的学生接触,但没想到会教课这么累人。

记得上次课没有这么累,但到时自己老了?

当然这不太可能。

翻了个身,平躺着凝视着天花板。

虽然这个建筑在二次出售时加固了整体结构,但是这种为结构性的天花板的老化与墙皮脱落就不算在内了。

油性漆,原子灰,墙板……

那些光鲜的装饰一层一层地被剥离,也就一些墙角有着哪些装饰的残余。

如果是刷上的油漆的话不会顺着墙流下来么,而如果是喷漆的画偶是如何避免涂料飞溅自己一脸的?

想着这些有的没的乱七八糟的维持自己思维活跃的伪命题,从床上翻下来。

现在是服务器使用的空闲期,基本上不会出现80%以上运算的需求。

将自己的个人电脑同步了10\%的服务器运算量,决定玩点网络游戏。

现在的游戏真是厉害,记得在边罗刹的时候,我电脑唯一能玩的是那种用命令提示符拼凑的游戏,最多也只是那种跟dos系统没什么区别的MUD游戏。

对我而言,就算现在的游戏做得再真实再华丽,也不会和当年一样有趣了。

会试着将任意的道具

记得那是刚接触mud时,会试着将任意的道具相互组合,总会得到意料之外的惊喜。

当然玩网络游戏纯粹是为了打发时间,我并不会将一款游戏当做“大型聊天室”的。

平京“这种采集任务加到游戏里纯粹是为了拉长游戏时间。”

一边弹手用鼠标在屏幕上点击关键道具,一边自言自语着。

平京“……为什么防沉迷系统会对一个成年人生效?我明明已经按照自己的身份证注册完了。”

看着自己已经被锁定了的经验槽,随手在网上查询到“必须要支付实体货币购买道具才能获得额外的每日经验”这样的信息。

怎么现在还会有如此让人不爽的设定存在?

算了,反正今天的日薪已经到账了,奢侈一把也没什么不好。

将今天的日薪全部打入游戏账户……好麻烦,这些点券金币都是什么莫名其妙的玩意。

随手买了几个经验道具,继续打发我无聊的深夜时间。

平京“恩?竟然有人在游戏里找我,这么罕见。”

……

等等,不对劲。

为什么我会和陌生人聊如此长的时间……

我记得我在游戏中根本没有和对方有任何互动,为啥会找上我。

还有对方声称会用7折帮我兑换游戏内的道具……

结果并没有。

之前还说得好好的,甚至还确认了详细的步骤,结果还是没有回应。

而且在划账之后就完全没有联系了……

等等,难道说……

我被骗了,被这种低级的方式?

怎么可能?明明游戏中到处都标着不要相信官方以外的任何销售信息……

我要静一静……

坏菜了

这可是我到时候三天的日薪的量就这么打水漂了,不对,打水漂还能听个响,这玩意连个屁都没有。

虽然这只是三天的量,但好歹也是北育的平均月薪了,我不甘心。

看来只能试着去用自己的能力去追溯了。

从游戏下手吧,虽然这个游戏很庞大且复杂,但是还有很多的部分使用Unix的老接口,当年虽然看着是很严密的防护措施,但是现在看来完全是不用泡一杯咖啡的时间就能破解的简单屏障。

好的,现在是进入运营商的管理界面了,虽然说理论上说这个层面上没有了那些用了超过三十年的老程序,但是我这里还有一份标准的密码测试包。

也并没有什么特别的技术含量,只是包含着那些“qwerty”,“Password1”之类的无脑密码而已。

当然,在这超过300个管理层级别的账号里,肯定有那种不动脑子设置密码的。

就这样不断地逆向追溯,还是在一层防护上卡住了。

平京“为什么这个账号会挂在这个网络系统上,还用和这个网络性质相对的洋葱网络外挂……”

这到底是何方神圣啊,能做带这种地步。

不对,是谁能够有权利在统一网络上刮一层多层匿名网络系统……

平京“虽然如此,还是不甘心啊。”

虽然会导致额外的支出,但也只能去找美香紙了。

平京“美香紙,在么?”

美香紙“老板有需求我就会出现,有问题么?”

平京“你帮我去追溯一个账号的信息,所在地密码之类的哪个都行。”

美香紙“那可不是完全合法的行为啊,老板。”

平京“反正你的老板我已经被坑了一把钱了,在当一次冤大头也没有区别。”

平京“成了到时候打给你13000当做报酬,是个人资产转移,不包税就是了。”

美香紙“其实那些合同之类的都无所谓,老板的所有发言都没有违约过不是么,只要老板开价我就会帮忙的啦。”

平京“恩,我是追查到xxx.xxx.xxx.onionplus就不能往下追溯了。”

美香紙“洋葱网络的破译对老板来说不应该是没有什么问题的么?”

平京“你不仔细看看,这个域名之前的指针是指向统一网络的,这玩意我可没有破解的经验,我也不知道这玩意是怎么和统一网络融洽的运行的。”

美香紙“恩,所以你就推给我了么……这玩意我还得去从其我的那台老古董,我要加价,老板”

平京“我就猜到了,2000不能更多”

美香紙“诶?我想加到16000的。”

平京“不行”

美香紙“诶,为啥啊,一千块也不给我更多。”

平京“因为那是风险收益的阈值,高一点我宁愿选择放弃。我可不相信什么神灵,甚至我还想隔扣你的报酬。”

美香紙“因为你口中的墨菲定律?”

平京“是的”

美香紙“那样你可不会找到女朋友的,活该老板到现在都是光棍。”

平京“……我不想和你说话,很多时候。”

美香紙“然而对话还在进行,不是么?”

平京“所以,就15000了,我已经把钱打给你了。”

美香紙“我可不保证能够成功啊,老板”

平京“我所谓,反正这个价格是包含失败的成本的。”

……

美香紙“虽然说很有挑战,但也难不住我。”

平京“动作很快嘛,怎么样了?”

美香紙“这样如同跳骚不断变换的端口连接虽然说是难以持续追踪的,但是可惜我已经看破了其中的伪随机规律,直接强行模拟了对方的连接模式进行持续的渗透,然后得到了真正的ip地址,还是ipv4的,真的是很罕见。”

平京“你给我等等,你的意思是你找出了椭圆曲线密码的几个关键标量然后直接从曲线上把密码找出来么……”

美香紙“本质上说就是这样”

平京“也就是说就算这样你也只是找到了地址,没有其他的了?”

美香紙“没有了”

平京“……那你就把地址告诉我”

美香紙“220.1xx.x.xxx”

平京“等等……这个ip看着很眼熟。”

美香紙“没错啊,我还确认过,就在你家楼上……”

平京“你确定这不是个肉鸡ip?”

美香紙“我怎么会在一个肉鸡上停滞不前,这当然是那个源信息的发出者的ip了。”

平京“我就信了你的邪,那就当你完成工作了。”

美香紙 “谢老板开明。”

关上聊天软件,将自己从工作台前推出来,转椅也因此碰到了后面的障碍物。
