\chapter[second chapter]{分歧的章节}

包含六层能够住人的也就五间,有三件事别人挂名的空房,还有一个我记得老人是自然过世了,也没有看到有什么子女过来继承遗产。应该也不会有人要这里的房子吧,毕竟没有什么伴随经济增值项目。自从去年夏天那个老人离开之后就再也没有回来,我知道老人死去也是再后来确认产权的时候知道的。

还有一个,我记得是长期有人居住的,但是我却从来没有看到甚至听到门打开的声音,从外面看那件住户的阳台一直都保持着灾难之后的样子,在阳台边上的碎玻璃就从来没有改变过位置。外面看也是如同没有通电一样。在这自立建筑生活超过一年的时间里,在公摊面积理我就没见过那个神秘的七层住户。

平京“这个世界真是小到让人讨厌”

顺着楼梯走道七层,门口已经看上去很长时间没有打扫了,就连门把都有一层明显可见的灰尘,很难让人联想到这里有人居住。

随意观察了一下,们的旁边没有任何类似于命令的装置,就算向我的门口的那种敲门器都没有,只能直接用指关节敲打门房。

平京“有人么?”

但是没有任何回答,在又试了几次后就用手锤房门,用更高的音量喊。

平京“亲人有人在里面么,这里是住在六层的平京,如果有人请回答。”

然而还是没有任何反应。

我就不信你的邪,今天非要把这个门打开不可。

……

没想带根本没有门锁,外面看着这么坚固的防盗门就像普通的木门那样被推开了。

平京“私自进来失礼了。”

虽然这么说,但是里面还是没有回应……这不太正常。我不是过度信任美香紙,只是因为她给了我全面的追溯材料,我能够确认这次她的工作没有糊弄我就是了。

当然,这更加可疑了。

??“谁啊,竟然连门也不敲”

平京“恩,看来果然有人在里面啊,没白费我一番功夫。”

??“话说你是谁啊,私自进别人的”

平京“我是谁并不重要,更关键的是你需要还钱。”

??“我和你认识么?我认为我应该报警了。”

平京“现在还这么说么,骗子,该报警的人是我。只是我很清楚就算报警了那些条子也不会来这里的。”

??“所以说你到底是谁。”

因为这个房间用遮光窗帘拉上,所以唯一的光源是这个人使用的屏幕。

不过不可思议的是其屏幕是用阴极射线管做为输出源,所以整个房间都是以屏幕上的主演颜色作为基调照明的。

对方终于转过身,我也因此得以看到对方的相貌。

对方是个女性,看起来有一些不健康的纤细,头发虽然不长但也有一些的打绺与卷曲。穿着的是陈整蚕的夏季校服

然后在外面还套着一件看起来比较厚实而又没有任何特色的红色运动服,唯一能称得上的装饰是从袖子连到肩膀的几条白线。

平京“恩,想不到竟然还是学生啊,把名字报上来。”

??“都来到这里了还不知道我的名字你不觉得可悲么?”

平京“再可悲你也想不到我是一名教师,恩,这样就好办了。”

??“什么?你在说什么?”

平京“喂,是周校长么,我啊,是我平京。”

??“你在干什么啊!”

平京“我就直说吧,是不是有一个拒绝来学校的学生?恩,我认为是高二的女生。”

平京“哦哦,还真有啊Спасибо,那这个学生就由我管了。什么?没问题,毕业班的学生就不用你这么操心了,有我在你还不放心?”

挂断电话,现在的我充满了主导权。

平京“陈天琪同学,虽然现在你已经不在义务教育阶段了,但是学校发现因为你没有相应的监护人,所以就由我来管理了。”

陈天琪“……我选择服”

平京“好了,你就老实把网上骗的钱还回,当然我是指昨天的那笔。”

陈天琪“原来是这样啊……”

陈天琪“真是遗憾,就算已拥有所谓确凿的证据,也从我这里拿不到一分钱。”

平京“你这话是什么意思?”

陈天琪“你只是被监护人财产的代管人,而非财产的所有人,因此,平京先生是不能处理被监护人的财产的。”

平京“但那时我的东西啊!”

陈天琪“只可惜你所谓的证据是非法获取的,而‘非法获取的物证所采用的违法证据排除规则’的存在会让你所有的证词站不住脚。”

平京“也不知道是谁非法侵入并更改统一网络的”

陈天琪“这句话我原句奉还,当然你认为就南区的民事法院会知道统一网络的一分一毫?”

平京“我很抱歉,我今天是作为一个长者给你们这样讲。我不是一个中国人,但是我见得太多了,我有这个必要告诉你们一点人生的经验。”

平京“你给我从明天开始乖乖上学去!”

……

恩,很好。

平京“恩……看来我私下作的工作也是有成效的……总之……我也得到了一些要求,吗……那叫什么来着……第一次模拟。”

平京“我记得是这个名字……总之,在前一个月的学习一定是基于一次模拟的教育大纲施行的,你们有什么问题尽管问。”

然后,那个新来的老学生果不其然发话了。

陈天琪“我能够免去补习么,我想我进行测试的话标准什么的肯定不会成为问题。”

平京“很好,下一个问题。”

陈天琪“I am angry!”
 
平京“好吧,那让我们先从多元微积分开始作为基础吧……毕竟我认为这只是单变量微积分的拓展而已。你们只需要了解思路,无需进行繁复的计算。毕竟这玩意的计算量随变量数的增长使指数形式的。”

然后,就是我的念书时间。

可能有人会疑惑,为什么这些中学生需要接触这些曾经大学生都感到无比头疼的高等数学内容。

关于这点,我只能说这帮大脑都没有完全发育的孩子,是肯定不能够“理解”这些内容的,只可能“运用”其中的方法。

换一种理解方式……就算一个硅板计算机的功能是图灵完备的,也不会理解$2\times 3$ 和 $3\times 2 $的区别。

计算机只会加法,当然这是由于其逻辑门原理的缺陷造成的。

而这些乳臭未干的小鬼们能够使用这些集合论下的一个个进化总结也是类似的。

就算他们能够灵活的运用$\frac{\partial f}{\partial x}$进行梯度的运算,他们也是不会理解为什么人们会认为偏导数是不完备的。

当然,现在的物理学也基本上没有人能够理解工具的本质了,不过这也和这帮小鬼们的关系就更小了。

他们只是在幼年期进行了不同于一般的感官刺激……具体怎么样的刺激我也没有体验过,不过相关的论文我还是有所了解的。一个人的个人存在本质上是通过感官的刺激与身体的反馈双方向交互而形成的。

当然……我读完了以后只是感觉那是一个被点明了的废话。

反正不管是脑后插管还是戴铝箔帽子,很多进行重组后的数学知识就这么灌输到了那些还没有完全发育的海马体里。

总之……知道了理论,去对症下药也会容易很多

对他们教学,只用表述问题特征和相应的解决方法就可以了……高度模式化,也没有什么新奇的思路可言。

如果现在大家都用的是过去被称为高明的方法解决问题,这种方法还能够称之为“高明”的话就实在是太可笑了。

这也只是“常识”的一部分了,已经融入了大家的知识之中难以抹去了。

总之,今天是过了一遍从多元函数到拉格朗日乘数的内容,不过我估计关于Lagrange multiplier的内容肯定还得花一天才能够给他们讲清楚。

方法并不复杂,但是运用描述困难。

而拥有这种特征的理论数不胜数,我想这是那种刺激下所了解的只是难以触及的。

不软如何,我以极高的压缩率将一天的课程用一个上午就讲完了。

虽然说现在陈天琪名义上是我的被监护人,但是她一下课就跑得没影了。

当然,我也不在乎这些,多一事总是比少一事要来的麻烦。

回去的路上,是没有什么店铺的。或者说真正的那些能做到店门大开的店铺也只会存在于治安相对稳定的北育市中心。

虽然我这边并不安全,但是我确认是没有人会对我造成威胁的。

当然这不是因为我有多么强大的能力,而是因为我知道如何伪装自己。

如果说所有的行为都有代价的话……对于我而言更多的是阿尔卡蒂奥被奥雷利亚诺推翻了……

在自己重复着曾经已经实施过不可数次数的不精准分析后,我让自己放空意识行走。

然后,我走到了这家店前。

周围都没有基座能够称之为”完整“的建筑,但是一些只有一个点是个例外。

虽然建筑的上半部分已经坍塌,外墙的保卫玻璃也没有留下任何一块完整的在上面,但是三层以下以框架结构搭建的部分却非常完整。

整个检出就挨着已经废弃了的主干线。里面已基本上长满了杂草,大部分的墙壁上的油漆已经脱落,能看到冷灰色的腻子和发白的混凝土与少量折出来的钢筋,地上也散布着各种残渣,从高出掉落的餐车到天井的承重梁。

而在这混乱的边缘,有一家店铺却依旧没有打样。

里面就像是另一个世界,整个店铺用米黄色的墙纸和深色木头做防护的装饰,却很别致的用汞离子荧光灯作为照明光源。

里面也不卖什么值钱或者生活的必需品。

当然,对我来说基本上也算是必需品。

那是一家咖啡店。

准确的来说是一家pub,但是据我的询问老板是不会进新酒品的,而咖啡就算是再偏门的品种都会及时补货。

就结果而言,这家pub是没有任何酒精而却有很多的咖啡因。

我当时随口说

“你干脆改店名为咖啡馆吧”

后来,我记得连前缀都改了,现在是叫做“胜利咖啡”

以前我记得叫做……

已经记不起来了,不过也无所谓了。

我直接推开了咖啡馆的门。

“欢迎光临”

虽然从功能上说是一个咖啡馆是毫无疑问的,但是将其带入二十一世纪初的文学作品中的咖啡馆是明显不合时宜的。

这里没有除了店主以外的任何工作人员。

而店主也比起待客之道更注重对咖啡的冲泡和店里的安全。

虽然我来的时候都不会有其他的客人,但是店长说还是有稳定的客源的——当然我也算在其中之一了。

店员“很久不见嘛,这次你是要美式咖啡么?像以往那样。”

平京“不用,这次我不要750cc的咖啡,我要两分的那种浓缩,当然是在这里,不用纸杯带走。”

店员“espresso是吧,要什么别的东西么”

平京“什么都不用加,连crema都不用”

店员“加入咖啡脂不会有任何多余的花费……”

平京“我不喜欢,所以不用。”

好像店员在期待我去将咖啡豆里面那如同糖浆的高能量胶体打在咖啡的顶层一样,虽然本质上说也不是什么有毒的东西,但是我并不认为这是一种好的选择。

店员“请等待十分钟。”

因为没有连续的客流,所以这里连热水都是来了客人现煮的。



